\documentclass[a4paper, 13pt, draft]{article}

\usepackage{graphicx}

% \usepackage{color}
\usepackage[dvipsnames]{color}
\usepackage{enumitem}

\definecolor{col1}{RGB}{204, 0, 0}
\definecolor{default}{RGB}{20, 20, 31}

\usepackage{multicol}
\setlength{\columnsep}{40pt}
\usepackage{geometry}
\usepackage{hyperref}

\newcommand{\timeperiod}[2]{%
    \textit{\small{#1 --- #2}}
}

\newcommand{\proglang}[1]{\subsection*{#1}}

\geometry{%
    a4paper,
    total={170mm,257mm},
    left=20mm,
    top=20mm,
}

\author{Norbert Logiewa}
\title{Norbert Logiewa}

\begin{document}

\color{default}


\begin{center}% 
    \Huge{%
	\color{col1}
	Norbert Logiewa
	\color{default}
	\\ \small{A Computer Science Student}  
    \end{center}

    \begin{table}[htpb]
	\centering
	\begin{tabular}{c}

	    0 742 8294 838 \\

	    \href{mailto:nl253@kent.ac.uk}{nl253@kent.ac.uk} \\

	    \href{https://github.com/nl253}{https://github.com/nl253} \\ \\

	    \textbf{Home address}\\ 
	    8 Conway Close, Colwyn Bay, North Wales \\ \\

	    \textbf{Term-time address} \\ 
	    7 Forty Acres Rd, Canterbury, England \\

	\end{tabular}
    \end{table}

    \section*{}

    \begin{multicols}{2}

	\subsection*{BSc Computer Science and AI}

    University of Kent, \small{Canterbury}} \\ \timeperiod{Oct 2016}{June 2020} (\textit{distinction for 1st year}) \\

    \begin{itemize}[leftmargin=*]
	\setlength\itemsep{0em}
    \item
	\textbf{Object Oriented Programming} \\
	\textit{Java, Testing with JUnit}
    \item
	\textbf{Software Engineering} \\
	\textit{stages, approaches, design patterns, UML}
    \item
	\textbf{Computer systems} \\
	\textit{operating systems, networking, hardware}
    \item
	\textbf{Algorithms and Data Structures} \\
	\textit{big O notation, analysis of algorithms}
    \item
	\textbf{Theory of Computation} \\
	\textit{natural deduction, finite state machines, regular expressions}
    \item
	\textbf{Human Computer Interaction} \\
	\textit{user experience, usability}
    \item
	\textbf{Web Development} \\
	\textit{HTML, CSS, SQL, JavaScript, JQuery, PHP, MVC CodeIgniter}
	\\ \\
\end{itemize}

\subsection*{Ysgol Eirias -- A-Levels} 
\small{Colwyn Bay, North Wales} \\ \timeperiod{Sept 2014}{June 2016} \\

\begin{itemize}		
    \item 

	\textbf{Psychology} A*

	\begin{itemize}		
	    \setlength\itemsep{0em}
	\item Core Studies
	\item Cognitive Psychology
	\item Abnormal Psychology
	\item Forensic Psychology
	\item Scientific Methodology 

    \end{itemize}		

\item 
    \textbf{Sociology} A

    \begin{itemize}		
	\setlength\itemsep{0em}
    \item Criminology
    \item Equality - Gender and Health
    \item Sociology of Family
    \item Scientific Methodology
\end{itemize}		

\item 
    \textbf{Health and Social Care} (\textit{double}) & D*D* \\ \\

\end{itemize}		
\end{multicols}

\subsection*{Subway, \small{Llandudno, North Wales} \\\emph{Sandwich Artist}} \timeperiod{Sept 2014}{June 2014}

\begin{itemize}		
    \begin{multicols}{3}
    \item Working within a team
    \item Customer service
    \end{multicols}
\end{itemize}		

\section*{\color{col1} \Large{Programming} \color{default}} 

\proglang{Java}

Knowledge of core Java and the standard library

\begin{itemize}
    \setlength\itemsep{0em}
\item
    Functional Programming (\textit{lambda expressions, Stream API, Optionals})
\item
    Understanding Object Oriented design (\textit{inheritance, Classes, Interfaces, Generics})
\item
    Data structures, understanding how they scale --- knowing when to use which
\item
    Knowledge of the Java ecosystem and tools (\textit{Maven, Javadoc, IDE Intellij Idea etc.})
\end{itemize}

\pagebreak

\begin{multicols}{2}

    \section*{\color{col1}Scripting\color{default}}

    Experience writing automation and build scripts in different languages.
    Although I feel most comforable with Python and Bash, I have dealt with
    Node.js, TypeScript and VimScript.

    \proglang{Python}

    \begin{itemize}[leftmargin=*]
	\setlength\itemsep{12pt}
    \item
	Tabular data manipulation using Pandas\textbf{,} visualization using
	Seaborn
    \item
	Using SQLAlchemy (ORM) to communicate with databases
    \item
	Templating language Jinja
    \item
	Writing build and utility scripts
    \item
	Command line interfaces using argparse, prompt\_toolkit and docopt
\end{itemize}

\section*{\color{col1} Computer Systems \color{default}}

Understanding of core concepts such as memory management, CPU time,
threading, multiprocessing, scheduling, subprocess etc. \\

\subsection*{GNU / Linux}

\begin{itemize}[leftmargin=*]
    \setlength\itemsep{0em}
\item Solid knowledge of GNU Coreutils
\item Bash
\item Vim
\item Experience using UNIX command line tools \\
    curl, ssh, rsync \dots
\end{itemize}

\subsection*{Databases}

\begin{itemize}[leftmargin=*]
    \setlength\itemsep{0em}
\item Communicating with databases using Python \\ (\textit{SQLAlchemy ORM})
\item SQL \\ 
\end{itemize}
\end{multicols}

\section*{\color{col1} Additional Skills \color{default}}

\begin{itemize}
    \begin{multicols}{2}
    \item Git (\textit{Version Control System})
    \item \LaTeX
    \end{multicols}
\end{itemize}

\begin{multicols}{2}

    \subsection*{Peronal Projects}

    \begin{itemize}
	\item \href{https://github.com/nl253/FlaskyBlog}{Python Blog \\ https://github.com/nl253/FlaskyBlog}
	\item \href{https://github.com/nl253/SQLiteREPL}{Python SQLite REPL  \\ https://github.com/nl253/SQLiteREPL}
	\item \href{https://github.com/nl253/Scraper}{Simple web crawler \\ https://github.com/nl253/Scraper}\\ \\ \\
    \end{itemize}

    \subsection*{Personal Interests}

    \begin{itemize}
	\item Fitness
	\item Programming Languages (such as \textit{Haskell}) and operating systems in particular
	\item Keeping up to date with open source projects on GitHub \\ \\ \\
    \end{itemize} 

\end{multicols}

\hline

\section*{References}

\begin{multicols}{2}
    \subsubsection*{Tutor}
    Olaf Chitil \\
    \href{mailto:O.Chitil@kent.ac.uk}{O.Chitil@kent.ac.uk} \\
    \subsubsection*{Employer}
    Elizabeth Gilmore-Jones \\
    \href{mailto:jonesegbiz@yahoo.com}{jonesegbiz@yahoo.com} \\

\end{multicols}
\end{center}

\end{document}
